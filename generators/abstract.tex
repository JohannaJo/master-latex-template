\pagenumbering{roman}
%\cite{schlichtkrull2018modeling, vashishth2019composition} (for neural networks citation)
\begin{abstract} 

\noindent Knowledge graphs (KGs) have risen in both size and use over the past few years, and there are a range of approaches for evaluating information in them. Symbolic approaches, such as rule-based machine learning, offer an explainable way to determine the appropriateness of a new fact based on rules mined from the KG in question. Some of the most successful approaches for fact prediction in KGs today are knowledge graph embeddings (KGEs) that use deep neural networks, however, these lack the explainability of symbolic approaches. We would like to see how the extension of a KG using KGEs affects the rules mined from the KG. A set of rules is mined from a KG and compared with another set mined from an \textit{extended} version of said KG. The experiments examine three classical KGEs: TransE, DistMult, and ComplEx, and use the rule mining algorithm AMIE3. AMIE3 is treated as a black box during the experiment, as the study only evaluates factors playing a role in the KG-extension process, one of which is the choice of KGE. The experiments show that there can be considerable discrepancies in the rules mined based on the choice of KGE and that TransE leads to a substantial amount of nonsensical rules being mined.

\end{abstract}

\renewcommand{\abstractname}{Acknowledgements}
\begin{abstract}
    First and foremost, I would like to thank my two supervisors, Ana and Ricardo. Ana was, from the start, a fantastic guide for me in the field of academic research and taught me a lot about writing and publishing work. Ricardo was always eager to answer and discuss questions, and his attention to detail taught me a lot. A great thank you to both of you. You have given me more support through this master's than I could have hoped for.
	
	I would like to thank my fellow master's students that battled with me over the two years. The workload is not so heavy when you are in it together. Thank you John Isak, Hans Martin, Knut, Halvor, Mathias, Magnus, Emir and Oda. You have been a fantastic group to work with, and I wish you all the best in your future work/studies.
	
	My final thanks goes out to family and friends, who have helped me by listening to my thoughts and reviewing the thesis. Complicated ideas become tangible and manageable when you put the correct words to them. Thank you for helping me find them.
	
	\vspace{1cm}
	\hspace*{\fill}Johanna Jøsang\\ 
	\hspace*{\fill} 01 June, 2022
\end{abstract}
\newpage