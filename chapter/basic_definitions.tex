\chapter{Basic Definitions}

%\newcommand{dltext}[1]{\centerline{\textsf{#1}\newline}}

\section{Knowledge Graphs}
There is no single agreed upon definition of \glspl{kg} \cite{bergman_2019, bonatti2019knowledge, ehrlinger2016towards}. Definitions and usages vary from specific technical proposals to more general descriptions. In this thesis we will use the more inclusive definition similar to the one proposed by Hogan et al. \cite{hogan2020knowledge}, where we view a \gls{kg} as \textit{a graph of data intended to capture the semantic connections within real world knowledge, where nodes represent relevant entities and edges represent relations between these entities}. The type of graph may vary, i.e. it may be simple, directed, etc. The concept of ``knowledge'' has been widely debated in epistemology, but here we will use it to mean descriptive knowledge, meaning facts that can stated. Knowledge can be simple statements, such as ``Leo is a cat'', or quantified statements such as ``at least one cat is black''. KGs are not expressive enough for quantified statements, where ontologies or rules would be more appropriate. Additional knowledge can be inferred from KGs through inductive or deductive methods. For example from a KG containing the information that ``Leo is a cat'' and ``cats are mammals'', one can deductively infer that ``Leo is a mammal''. If all cats mentioned in the knowledge graph like to eat fish, then one can inductively infer ``cats like to eat fish''.


%For example, the knowledge that 'Amy is the daughter of Bo, and Bo is a woman' can be represented by the knowledge graph in fig 1.\todo{Add figure}
\begin{figure}
\centering
\begin{tikzpicture}
    \node[shape=circle,draw=red] (L) at (0,0) {Leo};
    \node[shape=circle,draw=black] (C) at (3,0) {Cat};
    \node[shape=circle,draw=black] (M) at (1.5,3) {Mammal};

   % \path [->] (L) edge node[left] {$IsA$} (C);
   \draw [->] (L) -- (C);
   \draw [decoration={text along path,
    text={is a},text align={center}},decorate]  (L) -- (C);
    
    \draw [->] (C) -- (M);
   \draw [decoration={text along path,
    text={subclass},text align={center}},decorate]  (M) -- (C);
    
    \draw [dotted, ->] (L) -- (M);
   \draw [decoration={text along path,
    text={is a},text align={center}},decorate]  (L) -- (M);
    
  
\end{tikzpicture}

\caption{Example of a knowledge graph, where the dotted line represents a relationship that can be inductively inferred.} \label{fig:KGexample}
\end{figure}

\section{Ontology language}
A widely used formal language for expressing ontologies is the \gls{owl}. In OWL "Daughters are female" could be formally expressed as:

\centerline{\textsf{SubClassOf(Daughter Female)}}
Information expressed in OWL can be used to draw new conclusions. For example if we know that an individual \emph{Amy} is a daughter, then we can makes the same conclusions as earlier about Amy being female. In OWL, the fact that Amy is in the class of females can be expressed as:

\centerline{\textsf{ClassAssertion(Female amy)}}
The task of reaching such conclusions is called reasoning and the type of conclusions can be drawn is specified by the \gls{w3c}. It specifies the \emph{semantics} of OWL, but does not present algorithms for how to derive inferences in practice. Sound and complete reasoning in OWL is of high complexity \citeauthor{Krotzsch2012}. Therefore, when the standard was updated to OWL 2 in 2009, it introduced restricted sublanguages to address this problem. These sublanguages restrict expressivity in order to simplify the reasoning task. One of these languages is OWL 2 QL, which is based on a \gls{dl} language called DL-Lite. OWL 2 QL is intended as a language to enable easier queries to databases. The ontology language we will use is DL-Lite$_{\mathcal{R}, horn}^{\exists}$, which is a member of the DL-Lite family.