\chapter{Equivalence of DL-Lite$_{\mathcal{R}, horn}$ and propositional logic}

It will now be shown that any DL-Lite$_{\mathcal{R}, horn}$ formula is logically entailed iff its translation to PL also is entailed. We start by defining a function $\pi$ that maps DL-Lite$_{\mathcal{R}, horn}$ concepts and concept expressions to formulas in PL. Then we show that for a DL-Lite$_{\mathcal{R}, horn}$ ontology and formula, respectively $\Gamma$ and $\alpha$, $\Gamma \models \alpha$ iff $\pi(\Gamma) \models \pi(\alpha)$.

\section{Definition of $\pi$}
We start with the translation of concept expressions:
\begin{itemize}
    \item $\pi(\top) := \top$
    \item $\pi(A) := P_A$
    \item $\pi(C \sqcap D) := \pi(C) \wedge \pi(D)$
    \item $\pi(\exists .r\top) := $
\end{itemize}
 where $C$ and $D$ are concept expressions, $A$ is a concept name and $r$ is a role name. $P_A$ is a propositional symbol that corresponds to the concept name $A$. Furthermore, for an interpretation $\mathcal{I}$ that satisfies a DL-Lite$_{\mathcal{R}, horn}$ formula $\alpha$, the translation of the interpretation will also satisfy the translation of $\alpha$, ie:
 
 \[\mathcal{I} \models \alpha \; \Rightarrow \; \pi (\mathcal{I}) \models \pi(\alpha).\]
 
$\mathcal{I}$
If $\Gamma \models \alpha$, then $\alpha$ is satisfied for every model of $\Gamma$. This means that for some interpretation $\mathcal{I}$ such that $\mathcal{I} \models \Gamma$ then $\mathcal{I} \models \alpha$.
 
 