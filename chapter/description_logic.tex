\chapter{Description Logic}

%\newcommand{dltext}[1]{\centerline{\textsf{#1}\newline}}

\section{Motivation behind DLs}
\gls{dls} are a family of languages used in knowledge representation and reasoning. The name represents two central aspects to this language group: \emph{description}, formal expression of knowledge, and  \emph{logic}, for it's logic-based semantics. DLs are used to represent domain knowledge in a well-structured and easily interpretable way. Domain knowledge is seperated into two components in DL, a \emph{terminological} part, called a TBox, and an \emph{assertional} part, called an ABox. The TBox represents knowledge about the structure of the domain, while the ABox has knowledge about specific instances. For example the fact that \emph{cats are  mammals} would be a TBox statement, while \emph{Leo is a cat} would be an ABox statement, as here we are making an assertion about the individual Leo. The combination of a TBox and an ABox is called a \emph{knowledge base} (KB).
As the semantics of DLs are logic-based it is clear when a statement is \emph{entailed} by a KB. For instance the two examples given above entail that Leo is a mammal. More importantly, this reasoning task can be automated in a DL KB. Reasoning tasks are performed with respect to the entire KB, which gives this language great power, but also comes with a computational cost. Therefore an important area of research has been to find DLs that strike a balance between expressiveness and the computational complexity of reasoning.


The two main criteria for a reasoner is that it is decidable and tractable (always correctly completed in a time that is polynomial with respect to the size of the KB).

\section{Syntax}
The description logic Attributive Concept Language with Complements ($\mathcal{ALC}$) is the basis for many DLs, and will be used to introduce the basic syntax and semantics commonly used in DLs.

In DL one works with three different types of ``building blocks'':
\begin{itemize}
    \item \textit{individuals} are specific concepts related to an instance,
    \item \textit{concepts} are a set of elements and can be viewed as unary predicates. The set a concept represents is called its \textit{extension},
    \item \textit{roles} can be viewed as binary predicates and represent relations between elements.
\end{itemize}
\textit{Concept names} and \textit{role names} are treated respectively as unary and binary predicates and are \textit{atomic concepts}. Other atomic concepts in $\mathcal{ALC}$ are $\top$ and $\bot$. In contrast, concepts and roles are \textit{complex concepts} (also called \textit{compound concepts}). A compound concept is constructed using at least one of the available operators in $\mathcal{ALC}$: $\neg$, $\sqcap$, $\sqcup$, $\forall$, $\exists$. Given that $C$ and $D$ are concepts and $R$ is a role, $\mathcal{ALC}$ allows the following complex concepts:
\begin{itemize}
    \item $\neg C$ (\textit{negation}) everything that is not in the extension of $C$,
    \item $C\sqcap D$ (\textit{conjunction}) the elements that are in the extension of both $C$ and $D$,
    \item $C\sqcup D$ (\textit{disjunction}) the elements that are in the extension of both $C$ and $D$, or either,
    \item $\forall R.C$ (\textit{universal restriction})  everything that is related by a $R$ role, is related to a $C$ concept,
    \item $\exists R.C$ (\textit{existential restriction}) a concept (at least one) that is related by a $R$ role, is related to a $C$ concept,
\end{itemize}

In $\mathcal{ALC}$ we allow:
\begin{itemize}
    \item ABox assertions: $C(a)$ and $R(a, b)$,
    \item TBox assertions: $C \sqsubseteq D$,
\end{itemize}
where where $a$ and $b$ are individuals, $C$ and $D$ are concepts, and $R$ is a role.

\section{Semantics}
In DLs, the semantics are defined by interpreting concepts as sets of individuals and roles as sets of pairs of individuals. The semantics of complex concepts and roles is defined in terms of the atomic concepts and roles.


We will now briefly cover the semantics of $\mathcal{ALC}$. An \emph{interpretation} $\mathcal{I} = (\Delta^{\mathcal{I}}, \cdot^{\mathcal{I}})$ consists of a non-empty set $\Delta^{\mathcal{I}}$ (the \emph{domain}) and a mapping $\cdot^{\mathcal{I}}$ that:
\begin{itemize}
    \item assigns every individual $a$ to an element $a^{\mathcal{I}}\in \Delta^{\mathcal{I}}$,
    \item assigns every concept name $A$ to $A^{\mathcal{I}}$, where $A^{\mathcal{I}}\subseteq \Delta^{\mathcal{I}}$,
    \item assigns every role name $r$ to  $r^{\mathcal{I}}$, where $r^{\mathcal{I}}\subseteq \Delta^{\mathcal{I}}\times \Delta^{\mathcal{I}}$,
\end{itemize}

such that the interpretation of $\top$, $\bot$ and a complex concept $C$ is defined inductively as:
\begin{itemize}
    \item $ \top^{\mathcal{I}} = \Delta^{\mathcal{I}} $,
    \item $ \bot^{\mathcal{I}} = \emptyset $,
    \item $(\neg C)^{\mathcal{I}} = \Delta^{\mathcal{I}} \backslash C^{\mathcal{I}} $,
    \item $ (C_{1} \sqcap C_{2})^{\mathcal{I}} = C_{1}^{\mathcal{I}} \cap C_{2}^{\mathcal{I}} $,
    \item $ (C_{1} \sqcup C_{2})^{\mathcal{I}} = C_{1}^{\mathcal{I}} \cup C_{2}^{\mathcal{I}} $,
    \item $ (\forall r.C)^{\mathcal{I}} = \{d\in \Delta ^{\mathcal{I}}\;|\; \forall d' \in C^{\mathcal{I}}, \; (d,d')\in r^{\mathcal{I}}\} $, \todo{perhaps differentiate between set theory quantifiers and DL quantifiers?}
    \item $ (\exists r.C)^{\mathcal{I}} = \{d\in \Delta ^{\mathcal{I}}\;|\; \exists d' \in C^{\mathcal{I}}, \; (d,d')\in r^{\mathcal{I}}\} $.
\end{itemize}


Interpretations can \emph{satisfy} a entities in $\mathcal{ALC}$, meaning that they fulfill the "requirements" to make the statement true. We write $\mathcal{I} \models x$ to say that $\mathcal{I}$ satisfies $x$. An interpretation $\mathcal{I}$ satisfies
\begin{itemize}
    \item $C(a)$ iff $a^{\mathcal{I}} \in C^{\mathcal{I}}$,
    \item $R(a,b)$ iff $(a^{\mathcal{I}},b^{\mathcal{I}})  \in R^{\mathcal{I}}$,
    \item $C \sqsubseteq D$ iff $C^{\mathcal{I}} \subseteq D^{\mathcal{I}}$,
    \item $R \sqsubseteq S$ iff $R^{\mathcal{I}}  \subseteq S^{\mathcal{I}}$.
    \item TBox $\mathcal{T}$ iff $\forall \Phi \in \mathcal{T}$, $\mathcal{I} \models \Phi$,
    \item ABox $\mathcal{A}$ iff $\forall \phi \in \mathcal{A}$, $\mathcal{I} \models \phi$,
    \item KB $\mathcal{K} = (\mathcal{T}, \mathcal{A})$ iff $\mathcal{I} \models \mathcal{T}$ and $\mathcal{I} \models \mathcal{A}$.

\end{itemize}


\section{Reasoning Problems}
Given a set of interpretations $\mathfrak{I}$ and a $\mathcal{ALC}$ TBox $\mathcal{T}$ such that $\forall \mathcal{I} \in \mathfrak{I}$, $\mathcal{I}\models \mathcal{T}$ the following questions are of interest:
\begin{itemize}
    \item Given a concept $C \in \mathcal{T}$, is $C$ satisfiable? I.e. is there an interpretation $\mathcal{I} \in \mathfrak{I}$ such that $\mathcal{I} \models C$?)
    \item Given two concepts $C, D \in \mathcal{T}$, is $C$ subsumed by $D$? I.e. is there an interpretation $\mathcal{I} \in \mathfrak{I}$ such that $C^\mathcal{I} \subseteq D^\mathcal{I}$?
    \item Given two concepts $C, D \in \mathcal{T}$, are $C$ and $D$ equivalent? I.e. is there an interpretation $\mathcal{I} \in \mathfrak{I}$ such that $C^\mathcal{I} \subseteq D^\mathcal{I}$ and $D^\mathcal{I} \subseteq C^\mathcal{I}$?
    \item Given two concepts $C, D \in \mathcal{T}$, are $C$ and $D$ disjoint? I.e. is there an interpretation $\mathcal{I} \in \mathfrak{I}$ such that $C^\mathcal{I} \cap D^\mathcal{I} = \emptyset$?
\end{itemize}

Similarly, for an $\mathcal{ALC}$ KB $\mathcal{K} = (\mathcal{T}, \mathcal{A})$ we are interested in the questions:
\begin{itemize}
    \item Is there an interpretations that satisfies $\mathcal{K}$? I.e. does $\mathcal{K}$ have a model?
    \item Given a concept $C$ and an individual $a$, does $\mathcal{K}$ entail $C(a)$? I.e. is there a model $\mathcal{I}$ of $\mathcal{K}$ that satisfies $C(a)$?
    \item Given a concept $C$, find all individuals $a$ such that a model $\mathcal{I}$ of $\mathcal{K}$ satisfies $C(a)$. I.e. given a concept $C$, find all individuals $a$ entailed by $\mathcal{K}$.
\end{itemize}

The more operators one allows in a logic the more complicated the TBox becomes, and usually the complexity for reasoning in the language increases. See \href{http://www.cs.man.ac.uk/~ezolin/dl/}{\textbf{Complexity of reasoning in Description Logics}} for an interactive look at the complexity of different DLs \cite{zolin_2013}.

\section{Naming conventions}

As we have seen $\mathcal{ALC}$ is the \textit{Attribute Language with general Complement}. The $\mathcal{C}$ actually denotes an extension of a more restrictive language $\mathcal{AL}$. $\mathcal{ALC}$ extends $\mathcal{AL}$ by allowing complex concept negation. The labelling of DLs starts with one of the basic logics $\mathcal{AL}$ (attribute language), $\mathcal{EL}$ (existantial language) or $\mathcal{FL}$ (frame based description language), followed by the following possible extensions:
\begin{itemize}
    \item $\mathcal{H}$: Role hierarchies
    \item $\mathcal{R}$: Complex role hierarchies
    \item $\mathcal{E}$: Full existential qualification, i.e. existential restrictions man have fillers other than $\top$.
    \item $\mathcal{O}$: Closed classes
    \item $\mathcal{I}$: Inverted roles
    \item $\mathcal{N}$: Cardinality restrictions
    \item ... \todo{Should these \newline extensions be \newline explained more \newline in detail?}
\end{itemize}



\section{Common restricted languages}

\subsection{$\mathcal{EL}$}

\subsection{DL-Lite}

\subsection{$\mathcal{RL}$}