\chapter{Description Logic}

%\newcommand{dltext}[1]{\centerline{\textsf{#1}\newline}}

\section{Motivation behind DLs}
\gls{dls} are a family of languages used in knowledge representation and reasoning. The name represents two central aspects to this language group: \emph{description}, formal expression of knowledge, and  \emph{logic} for it's logic-based semantics. DLs are used to represent domain knowledge in a well-structured and easily interpretable way. Domain knowledge is seperated into two components in DL, a \emph{terminological} part, called a TBox, and an \emph{assertional} part, called an ABox. The TBox represents knowledge about the structure of the domain, while the ABox has knowledge about specific instances. For example the fact that \emph{cats are  mammals} would be a TBox statement, while \emph{Leo is a cat} would be an ABox statement, as here we are making an assertion about the individual Leo. The combination of a TBox and an ABox is called a \emph{knowledge base} (KB).
As the semantics of DLs are logic-based it is clear when a statement is \emph{entained} by a KB. For example the two examples given above entail that Leo is a mammal. More importantly, this reasoning task can be automated in a DL KB. Reasoning tasks are performed with respect to the entire KB, which gives this language great power, but also comes with a computational cost. Therefore an important area of research has been to find DLs that strike a balance between expressiveness and the computational complexity of reasoning.


The two main criteria for a reasoner is that it is decidable and tractable (always correctly completed in a time that is polynomial with respect to the size of the KB).

\section{Syntax}
In DL one works with three different types of "building blocks":
\begin{itemize}
        \item \textit{Individuals} are specific concepts related to an instance
    \item \textit{Concepts} are a set of elements and can be viewed as unary predicates
    \item \textit{Roles} can be viewed as binary predicates and represent relations between elements
\end{itemize}

The description logic Attributive Concept Language with Complements (ALC) is the basis for many DLs, and will be used to introduce the basic syntax commonly used in DLs.



\section{Semantics}

