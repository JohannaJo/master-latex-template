\chapter{Learning from Neural Networks}

%\newcommand{dltext}[1]{\centerline{\textsf{#1}\newline}}
In this chapter we will look at how a neural network can represent a Horn ontology, and how the rules of this ontology can be learned by querying the network.

\section{Propositional Logic}
We define $V$ to be a finite set of boolean variables. A \emph{literal} over $V$ is either a variable $v \in V$ or the negation of a variable $v$, written as $\neg v$. A \emph{clause} is a disjunction ($\vee$) of literals. A \emph{formula} over $V$ is a conjunction ($\wedge$) of clauses over $V$.

A \emph{horn clause} is a clause where at most one literal is non-negated, and a \emph{Horn sentence} is a conjunction of Horn clauses. An implication can be represented as a Horn clause, where the antecedent of the implication is a conjunction of the the negated literals, and the consequent the non-negated literal. So if each rule in an ontology is represented as a horn clause, then a horn sentence containing all the rules would represent the entire ontology.

An \emph{interpretation} $\mathcal{I}$ over $V$ assignes truth values to all variables $v$ in $V$. A variable $v$ is \emph{satisfied} by $\mathcal{I}$ if $v \in \mathcal{I}$. If a variable is not in an interpretation $\mathcal{I}$, then it is said to be \emph{falsified} by $\mathcal{I}$. If a variable $v$ is falsified by $\mathcal{I}$, then the literal $\neg v$ is satisfied by $\mathcal{I}$. For a clause $c$ to be satisfied by an interpretation $\mathcal{I}$ at least one literal in $c$ needs to be satisfied by $\mathcal{I}$. For a formula $t$ to be satisfied by an interpretation $\mathcal{I}$, each clause in the formula needs to be satisfied by $\mathcal{I}$.

If a interpretation satisfies variable, literal, clause or formula $x$, one can write this as $\mathcal{I} \models x$. If an interpretation does not satisfy $x$ one writes $\mathcal{I} \not \models x $. If for every possible $\mathcal{I}$, $\mathcal{I} \models t$ implies $\mathcal{I} \models c$, then $t$ \emph{entails} $c$. This can be written as $t \models c$.


\section{Learning Framework}
We want to formally define the \empph{learning framework}. By \emph{learning} we mean the process of acquiring desired knowledge in a practical and machine-processable format. As input to the learner we give \emph{examples}, which are data points that encapsulate that knowledge. Formaly, a \emph{learning framework} $F$ is defines as a pair $(E, H)$ where
\begin{itemize}
    \item $E$ is a set of all examples,
    \item $H$ is the \emph{hypothesis space}.
\end{itemize}
In this situation $H$ is the set of all formulas in propositional logic, and $E$ is the set of all interpretations over $V$. The learning framework $F$ is said to be \emph{Horn} if $H$ is restricted to the set of all Horn formulas. An example $e \in E$ is a \emph{positive example} for a hypothesis $h \in H$ iff $e \models h$, and a negative example if $e \not \models h$. The target $t$ is a fixed element in $H$ which one in the learning process wants to identify. For $t, h \in H$ a \emph{counterexample} is an example showing that $h \not \equiv t$. A \emph{positive counterexample} $e$ is such that $e \models t$ and $e \not \models h$, while is $e$ is a \emph{negative counterexample} $e\not \models t$ and $e \models h$.

We follow the approach proposed by Angluin \cite{DBLP:journals/ml/AngluinFP92}, and assume that the learner has access to an oracle that can answer certain types of queries about the target $t$. Here we use two types of oracles:
\begin{itemize}
    \item A \emph{Membership oracle} $MO_{E, H}$ takes as input an interpretation $\mathcal{I}$, and outputs `yes' if $\mathcal{I} \models t$ and `no' otherwise.
    \item An \emph{Equivalence oracle} $EO_{E,H}$ is a function that takes as input a hypothesis $h$ and outputs `yes' if $h \equiv t$, otherwise it outputs a counterexample for $t$ and $h$. This counterexample can be either positive or negative.
\end{itemize}
A query to $MO_{E, H}$ is called a \emph{membership query}, and a query to $EO_{E, h}$ is called an \emph{equivalence query}.

A learning framework $F_{E, H}$ is exactly learnable if there is a deterministic algorithm $A$ such that for every possible $t\in H$, it eventually halts and outputs some hypothesis $h \in H$ where $h \equiv t$. In this scenario $A$ takes as input the set of variables $V$ over which $t$ is formulated and poses membership and equivalence queries, before finally outputting a hypothesis $h \in H$ where $h \equiv t$. $F$ is \emph{exactly learnable in polynomial time} if the the number of steps required by $A$ to find an equivalent hypothesis is bounded by the size of the target and the maximum length of any counterexample encountered.

\todo[inline]{Should the definition of a \emph{safe} learning framework be included?}

\section{Neural Networks}
In this work neural networks we will treat neural networks as a restricted version of traditional neural networks. Here a neural network can be understood as a way of representing a target formula $t$. The input to a neural network model $N$ is a $|V|$ dimensional vector with all its values in the range $\{0, 1\}$. This input represents an interpretation $\mathcal{I}$, where 0 and 1 denote the truth values assigned to the variables in $V$. More specifically, by $\overrightarrow{\mathcal{I}}$ we denote the vector in the $|V|$ dimensional space where each entry at position $i$ is 1 if variable $v_i \in V$ is in the interpretation $\mathcal{I}$, and 0 otherwise. So $N$ is a function which takes as input an interpretation $\mathcal{I}$ represented as a vector and outputs the satisfiability of the target formula $t$ under $\mathcal{I}$.
To train the neural network a dataset of the format $(\overrightarrow{\mathcal{I}}, l)$ is used. $l$ is either 0 or 1, indicating whether or not the interpretation $\mathcal{I}$ satisfies the target $t$. For every neural network trained on such a dataset there is a formula $t_n$ such that $N(\overrightarrow{\mathcal{I}}) = 1$ iff $\mathcal{I} \models t_N$.

With a neural network $N$ as an alternative representation of a target formula $t_N$, we look at how querying the neural network can lead us to exactly identify the target. We do this by following Angluin's approach of posing queries to two kinds of oracles \cite{DBLP:journals/ml/AngluinFP92}. A \emph{membership oracle} takes as input an interpretation $\mathcal{I}$, and outputs `yes' if $\mathcal{I} \models t_N$ and `no' otherwise. An \emph{equivalence oracle} is a function that takes as input a hypothesis formula $h$ and outputs `yes' if $h \equiv t$, otherwise it outputs a counterexample for $t$ and $h$. This counterexample is an interpretation that either satisfies $t$ but not $h$, or vice versa.

\todo[inline]{Should the definition of exact learning be introduced here?}

\section{Extracting Horn Ontologies from Neural Networks}

Given a neural network $N$ that represents an unknown target formula $t_N$ in the form of a Horn sentence, we are interested in discovering this formula $t_N$. Since the formula is Horn we can employ Angluin's algorithm for learning Horn theories \cite{DBLP:journals/ml/AngluinFP92}. This algorithm is called HORN. Given a finite set of variables, HORN is guaranteed to exactly identify a target formulated as a Horn sentence in polynomial time. To learn the target the algorithm poses equivalence and membership queries. HORN terminates when an equivalence query returns `yes', meaning that the hypothesis and target are equivalent. Membership queries are used to update the hypothesis, which starts empty and to which clauses that are falsified by a negative counterexample are added. HORN is guaranteed to terminate in polynomial time on the size of the target and the number of variables ($|V|$) in consideration. Using this algorithm Horn ontologies can the extracted from a trained neural network.

\todo[inline]{How thoroughly should the process of extracting Horn sentences from a NN be described? Should the process of simulating the equivalence oracle be included?}