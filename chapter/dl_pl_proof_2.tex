\chapter{Equivalence of DL-Lite$_{\mathcal{R}, horn}$ and propositional logic}

\newtheorem{lemma}{Lemma}

It will now be shown that any DL-Lite$_{\mathcal{R}, horn}$ formula is logically entailed iff its translation to PL also is entailed. We start by defining a translation $T$ that maps DL-Lite$_{\mathcal{R}, horn}$ concepts and concept expressions to formulas in PL. Then we show that for a DL-Lite$_{\mathcal{R}, horn}$ ontology and formula, respectively $\Gamma$ and $\alpha$, $\Gamma \models \alpha$ iff $\Gamma^T \models \alpha^T$.

We start with the translation of concept expressions:
\begin{itemize}
    \item $\top^T := \top$
    \item $A^T := P_A$
    \item $(C \wedge D)^T := C^T \wedge D^T$
    \item ${\exists .r\top}^T := P_r$
\end{itemize}
 where $C$ and $D$ are concept expressions, $A$ is a concept name and $r$ is a role name. $P_A$ and $P_r$ are propositional symbols that respectively corresponds to the concept name $A$ and the existential restriction $\exists .r\top$. Furthermore, for an interpretation $\mathcal{I}$ that satisfies a DL-Lite$_{\mathcal{R}, horn}$ formula $\alpha$, the translation of the interpretation will also satisfy the translation of $\alpha$, ie:
 
 \[\mathcal{I} \models \alpha \; \Rightarrow \;  \mathcal{I}^T \models \alpha^T.\]
 
If $\Gamma \models \alpha$, then $\alpha$ is satisfied for every model of $\Gamma$. This means that for some interpretation $\mathcal{I}$ such that $\mathcal{I} \models \Gamma$ then $\mathcal{I} \models \alpha$.

The translation of the pointed interpretation $(\mathcal{I}, d)^T$ will be represented by the function $f$, which maps formula in propositional logic to truth values. Let $p$ be a formula in propositional logic. For every $p$ in the set of all propositional symbols, $f(p)=1$ iff $p\in \{P_a \; | \; d\in A^{\mathcal{I}}\}\cup\{P_r \; | \; d\in {\exists r.\top}^{\mathcal{I}}\}\cup \{P_{r^-} \; | \; d \in {\exists r^{-}.\top}^{\mathcal{I}}\}$, otherwise $f(p)=0$.
 
\begin{lemma}
Let $\mathcal{I}$ be an interpretation and $C$ be a DL-Lite$_{\mathcal{R}, horn}$ concept. Given a pointed interpretation $(\mathcal{I}, d)$, $d \in C^{\mathcal{I}}$ iff $(\mathcal{I}, d)^T \models C^T$.
\end{lemma}
\begin{proof}
$C$ must be of the form $C_1 \wedge ... \wedge C_n$ where each  $C_i$ must be either a concept name $A$, an existential restriction $\exists r.\top$ or an inverse existential restriction $\exists r^{-}.\top$. \newline
First we show that $(\mathcal{I}, d)^T \models C^T$ if $d\in C^\mathcal{I}$.
\newline
If $d \in C^{\mathcal{I}}$ then $d\in \bigcap^{n}_{i=1}{C_{i}}^{\mathcal{I}}$. Therefore $f$ satisfies $C^T$, since $C^T = \bigwedge_{i = 1}^{n} {C_i}^T$ and $\forall 1 \leq i \leq n$ $f({C_i}^T) = 1$, by the definition of $f$.\newline
Next we show that $d\in C^\mathcal{I}$ if $(\mathcal{I}, d)^T \models C^T$.
\newline
Recall that $C^T = \bigwedge_{i = 1}^{n} {C_i}^T$. Therefore if $(\mathcal{I}, d)^T \models C^T$ then  $(\mathcal{I}, d) ^T \models {C_{i}}^T \; \forall 1 \leq i \leq n$ by the semantic definition of $\wedge$. If $(\mathcal{I}, d)^T \models {C_i }^T$ then by the definition of $f$ this means that $d\in{C_i}^\mathcal{I}\; \forall 1 \leq i \leq n$. If $d\in \bigcap^{n}_{i=1}{C_{i}}^{\mathcal{I}}$ then $d \in C^\mathcal{I}$ by the semantics of DL-Lite$_{\mathcal{R}, horn}$. \footnote{$ (C_{1} \wedge C_{2})^{\mathcal{I}} = C_{1}^{\mathcal{I}} \cap C_{2}^{\mathcal{I}} $, therefore if $d\in {C_1}^\mathcal{I}$ and $d\in {C_2}^{\mathcal{I}}$ then $d\in {C_1}^\mathcal{I} \cap {C_2}^\mathcal{I}$ and thereby $d\in (C_1 \wedge C_2)^\mathcal{I}$.}
\end{proof}
